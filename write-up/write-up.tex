\documentclass[a4paper]{article}

%% Language and font encodings
\usepackage[english]{babel}
\usepackage[utf8x]{inputenc}
\usepackage[T1]{fontenc}

%% Sets page size and margins
\usepackage[a4paper,top=3cm,bottom=2cm,left=3cm,right=3cm,marginparwidth=1.75cm]{geometry}

%% Useful packages
\usepackage{listings}
\usepackage{algorithm}
\usepackage[noend]{algpseudocode}
\usepackage{float}
\usepackage{amsmath,bm,mathrsfs}
\usepackage{amsthm,amssymb}
\usepackage{graphicx}
%\usepackage[draft]{graphicx}
\usepackage[colorinlistoftodos]{todonotes}
\usepackage[colorlinks=true, allcolors=blue]{hyperref}

%% Operators
\newcommand{\R}{\mathcal{R}}
\newcommand{\RR}{\mathbb{R}}
\newcommand{\N}{\mathbb{N}}
\newcommand{\G}{\mathcal{G}}
\newcommand{\V}{\mathcal{V}}
\newcommand{\E}{\mathcal{E}}
\newcommand{\La}{\mathcal{L}}
\newcommand{\UR}{U_{\mathcal{R}}}
\newcommand{\vv}{\mathit{v}}
\renewcommand{\l}{\ell}

\theoremstyle{definition}
\newtheorem*{definition}{Definition}
\newtheorem*{thm}{Theorem}

\title{Notes}

\begin{document}
\maketitle

\newpage


\section{Things to fix:}
\begin{enumerate}
\item (fixed) Undefined function 'findpeaks' for input
arguments of type 'double'.

Error in mcsfb\_design\_filter\_bank\_no\_fourier
(line 31)
    [~, idx] = findpeaks(inverted); 
    
\item (fixed) Warning: Matrix is close to singular or badly scaled (Reconstruction using eigenvalue and eigenvectors)


\item filter design function - does not perform well

\end{enumerate}

\section{Notations}
Consider a weighted, undirected graph $\G = \{ \V, \E, W\}$, where $\V$ is the set of $N$ vertices, $\E$ is the set of edges and $W$ is the associated weighted adjacency matrix. Let $D$ be the diagonal matrix of vertex degrees. Denote $\La$ as the graph Laplacian, where $\La = D-W$. Since $\La$ is real, symmetric and semi-definite, we can diagonalize it as $\La =U\Lambda U^*$, where $\Lambda$ is the diagonal matrix of its real eigenvalues $\lambda_0, \lambda_1,\cdots, \lambda_{N-1}$ and the columns $u_0, u_1 \cdots u_{N-1}$ of $U$ are corresponding orthonormal eigenvectors of $\La$. We use $\UR$ to denote the submatrix formed by taking the columns of $U$ associated with the Laplacian eigenvalues indexed by $\mathcal{R} \subseteq \{0,1,\cdots,N-1\}$. And we use $U_{\mathcal{S},\mathcal{R}}$ to denote the submatrix formed by taking the rows of $U_{\mathcal{R}}$ associated with the vertices indexed by the set $\mathcal{S} \subseteq \{1,2,\cdots, N\}$. For $i \in \{1,\cdots,n\}$, $\bm{\delta_i} \in \RR^n$ denotes the $i$th column of the identity matrix $I \in \RR^{n\times n}.$
We consider graph signals $f \in \RR^N$ residing on a graph $\G$. A signal $f \in \RR^n$ defined on the nodes of the graph $\G$ is $\R$-concentrated if $f \in \text{span}(U_{\R})$. The graph Fourier transform of a signal is $\hat{f} = U*f$ and $\hat{g}(\La) f = U\hat{g}(\Lambda)U*f$ filters a graph signal by $\hat{g}(\cdot)$. Signal $r\in \RR^N$ is a random vector such that each entry is randomly draw from a standard gaussian distribution.

\medskip

\section{Definitions}
%\begin{definition}
%A signal $\bm{x} \in \RR^n$ defined on the nodes of the graph $\G$ is $k$-bandlimited with $k \in \N^+$ if $\bm{x} \in \text{span}(U_{\R})$ and $|\R| = k$.
%\end{definition}

\begin{definition}
A signal $\bm{f} \in \RR^n$ defined on the nodes of the graph $\G$ is $\R$-concentrated if $\bm{f} \in \text{span}(U_{\R})$.
\end{definition}

\begin{definition}
Let $\bm{p} \in \RR^{n}$ represent a sampling distribution on $\{1,2,\cdots n\}$. The graph weighted coherence of order $|\R| $ for the pair $(\G, \bm{p})$ is
$$ \vv^k_{\bm{p}} = \underset{1\leq i\leq n}{\text{ max }} \{\bm{p_i}^{-1/2} ||U_{\R}^T \bm{\delta_i} ||_2\}$$

\end{definition}

\section{Theorems}
\begin{thm}[Orthogonality Proof]

Let $\{\R_1, \R_2, \cdots, \R_M\}$ be $M$ partitions of the graph Laplacian eigenvalue indices $\{0, 1, \cdots,  N \}$ and $\{g_1, g_2, \cdots, g_M \}$ be filters defined on each of the $M$ bands. Then, the subspace spanned by $g_i(\La)$ is orthogonal to the subspace spanned by $g_j(\La)$ for $i \neq j$

\end{thm}

\begin{proof}


Notice that the $(i,j)$ entry of matrix $g(\La)$ can be represented as 

$$g(\La) (i,j) = [Ug(\Lambda)U^T] (i,j) = \sum_{k = 1}^{N} g(\lambda_k) U_k(i) U_k(j)$$

Now consider two filters $g_1$ and $g_2$. To show that two subspaces spanned are orthogonal, we show that the dot product of any two vectors is 0.
\begin{align*}
g_1(\La)(,k) \cdot g_2(\La)(,l)&=\ \  \sum_{j = 1}^{N} [\sum_{k = 1}^{N} g_1(\lambda_k) U_k(i) U_k(j) )( \sum_{l = 1}^{N} g_2(\lambda_l) U_l(i) U_l(j)) ] \\
& = \sum_{k = 1}^{N} \sum_{l = 1}^{N} g_1(\lambda_k) g_2(\lambda_l)  U_k(i) U_l(i) ( \sum_{j = 1}^{N}  U_k(j)  U_l(j))  \\
& = \sum_{k = 1}^{N} g_1(\lambda_k) g_2(\lambda_k)  U_k(i) U_k(i) ( \sum_{j = 1}^{N}  U_k(j)  U_k(j)) 
\end{align*}

Notice that $g_1(\lambda_k) g_2(\lambda_k)$ is always equal to $0$, since $g_1$ and $g_2$ are defined on disjoint sets of Laplacian eigenvalues.   

\end{proof}



%\begin{proof}

%$$g_1(\La) = Ug_1(\Lambda)U^T$$

%$$g_2(\La) = Ug_2(\Lambda)U^T$$

%Notice that $g_1(\lambda_k) g_2(\lambda_k)$ is always equal to $0$, since $g_1$ and $g_2$ are defined on disjoint sets of Laplacian eigenvalues.  
%\end{proof}
%\end{comment}


\newpage

\section{Abstract}
     We investigate an $M$-channel critically sampled filter bank for graph signals where each of the $M$ filters is supported on a different subband of the graph Laplacian spectrum. For analysis, on each subband, the graph signal is filtered using efficient polynomial approximation methods and then downsampled on a corresponding set of vertices chosen via non-uniform random sampling. We use an efficient decoder derived from the non-uniform sampling distribution to reconstruct the filtered signals on each band from their samples. We leverage an approximation of the spectral density function of the graph Laplacian to both approximate the number of eigenvalues in each band and design the filter bank to be more amenable to polynomial approximation, in order to reduce the resulting reconstruction error. We empirically explore the joint vertex-frequency localization of the dictionary atoms, the sparsity of the analysis coefficients, and the ability of the proposed transform to compress piecewise-smooth graph signals. 



\section{Filter Bank Design}

In this section, we construct a scalable $M$-channel critically sampled filter bank. The ideal filter bank design requires expensive full eigendecompositions, which takes $\Theta(N^3)$ operations with regard to the size of the graph. In order to circumvent these expensive computations, we approximate the spectral density function of the graph Laplacian and then adopt it to approximate the number of eigenvalues in each band and to properly design the the filter bank. The filters are then applied using the polynomial approximation methods discussed in \cite{hammond2011wavelets, shuman_DCOSS_2011}. 

\subsection{Approximate Spectral Density Function}

Various methods have been proposed by past literature to approximate the spectral density functions, including the spectrum-based warping function method \cite{hammond2011wavelets,shuman_DCOSS_2011} and the trace approximation method \cite{approximating spectral densities of large matrices}, which we will use in our implementation. We give a brief overview of this method in this section.

Given a filter $\hat{g}(\cdot)$, we notice that $\hat{g}(\La)$ is a diagonal matrix, where $\hat{g}(\La)_{i,i} = 1$ if the $i$th eigenvalues is covered by the filter and $\hat{g}(\La)_{i,i} = 0$, otherwise. By computing the trace of $\hat{g}(\La)$, which is the sum of the diagonal, we can obtain the number of eigenvalues in the band specified by $\hat{g}(\cdot)$. This method scales linearly with regard to the size of the graphof $\Theta(N)$ time complexity. In practice, the trace can be approximated using the following fact:

$$tr(\hat{g}(\La)) = E[r^T\hat{g}(\La)r] \propto \frac{1}{I}\sum_{i =1}^{I} (r^{i})^T\hat{g}(\La)r^{i}$$


\subsection{Fast Filtering}

Recall that $\hat{g}(\cdot)$ filters a signal $f$ by $\hat{g}(\La) f = U\hat{g}(\Lambda)U*f$, where $\hat{g}(\Lambda) = (\hat{g}(\lambda_1), \cdots, \hat{g}(\lambda_n))^T.$  To avoid the computation of $U$ and $\Lambda$ in the filtering process, one common approach is to approximate the exact filter $\hat{g}(\cdot)$ by a polynomial function:

$$\hat{g} = \sum_{k = 0}^{\infty} c_kT_k$$

In our implementation, we allow two different polynoial approximation methods: Chebyshev or Jackson-Chebyshev approximations. For example, the truncated Chebyshev polynomial approximation can be represented as 

$$\hat{g}(\La)f = \frac{1}{2}c_0f + \sum_{k=1}^{\infty} c_k \bar{T}_k(\La)f \propto \frac{1}{2}c_0f + \sum_{k=1}^{K} c_k \bar{T}_k(\La)f$$ and each $\bar{T}_k(\La)f$ can be efficiently computed recursively from $\bar{T}_{k-1}(\La)f$ and $\bar{T}_{k-2}(\La)f$. We direct readers to \cite{Druskin and Knizhnerman} for more information on polynomial approximation and \cite{fast filtering} for more fast filtering techniques.


For the scalable filter bank, instead of storing the exact filters, we store the coefficients of Chebyshev or Jackson-Chebyshev polynomial approximations and the filter will be applied to any given signal accordingly.  \cite{shuman2013spectrum}.


\subsubsection{Adjust Band Ends to Minimize Approximation Error}
When selecting the band ends for each of the $M$ ideal filters, we consider two factors: spectrum-adaptation and band spacing method. In our implementation, the filter bank can be either spectrum-adapted or not, and it can be either evenly spaced or logarithmic spaced. All four possibilities are illustrated in the figure below:

\begin{figure}[h]
\centering
\includegraphics[width = 5cm]{filter_bank_1}
\includegraphics[width = 5cm]{filter_bank_2}

\includegraphics[width = 5cm]{filter_bank_4}
\includegraphics[width = 5cm]{filter_bank_3}


\caption{(a) is a spectrum-adapted and logarithmic spaced filter bank of 5 filters. Notice that this filter bank takes the density of Laplacian eigenvalues into account. The 4th filter covers half of the eigenvalues, the 3rd filter covers 1/4 of the eigenvalues, and each of the first two filters covers 1/8 of the eigenvalues.(b) stands for a spectrum-adapted but evenly spaced filter bank. Each of the five filters covers approximately the same number of eigenvalues. (c) and (d) are not spectrum-adapted. Notice that in this case the band ends only depend on the width of the spectrum.}
\end{figure}

Without a full knowlewdge of the spectrum of the graph Laplacian, we cannot pick the band ends the way as we do for ideal filter banks. While satisfying the spectrum-adaptation and spacing requirements, we shift our initial choices of band ends in a small interval in order to reduce the resulting reconstruction error.

Consider the exact filter $\hat{g}$ and the polynomial approximated filter $\hat{h}(\La)$. If $K$ is the order of the polynomial approximation, the filtering error is 
$$||\hat{g}(\La) - \hat{h}(\La)|| = \text{max}_{l = 0,\cdots,n} |g(\lambda_l) - h(\lambda_l)| \leq \text{sup}_{\lambda \in [0, \lambda_{max}]} |g(\lambda) - h(\lambda)| = \Omega(\rho^{-K})$$
Notice that for Chebyshev and Jackson-Chebyshev approximations, the filtering error is always clustered around band boundaries. If the boundaries are set in spectral gaps, i.e. there are no or very few eigenvalues around it, the approximated filters will not affect the filtering accuracy too much, thus leading to a small reconstruction error. 

Our approach to locate the spectral gaps is simply to find the minimum of the probability distribution of Laplacian eigenvalues over a specified interval. Let the $w$ be the width of the adjustment interval. For an initial choice of the band end $m \in [\tau, \tau+w]$, we intend to find $\tau$ such that the cumulative density function $F(\tau+w)-F(w)$ is minimized. That is, we want to minimize the probability of eigenvalues existing in the interval $[\tau, \tau+w]$. When $w$ is small, it is equivalent to find the $\tau$ such that the probability distribution of the Laplacian eigenvalues $p(\tau)$ is minimized.



{\color{blue}
\section{Non-Uniform Sampling and Reconstruction}
\begin{itemize}
\item Estimating number of measurements for each band
\item Review of non-uniform random sampling literature
\item Reconstruction - emphasize change from only lowpass bands
\item {\color{red} Solving the reconstruction equation efficiently and robustly}
\end{itemize}
}


\begin{thm}[\cite{puy}, Theorem 3.2]
Consider any graph $\G = \{\V, \E, W\}$ with its Laplacian $\La = U\Lambda U^*$. Notice that $\La$ is real, symmetric and positive semi-definite and thus the columns of U are orthonormal and its eigenvalues are non-negative. We propose the following sampling procedure:

\begin{enumerate}

\item Sampling Distribution. Let $\R \subseteq \{0,1,\cdots, n-1\}$. Define $\bm{p} \in \RR^n$ as the sampling distribution on vertices $\{1,2, \cdots n\}$  such that 

\[\bm{p_i} = \frac{||U^T_\R\bm{\delta_i}||^2_2}{|\R|} \text{, for } i = 1,2 ,\cdots, n. \tag{1}\]  
The sampling distribution $\bm{p}$ minimizes the graph weighted coherence. We associate the matrix $P = \text{diag}(\bm{p}) \in \RR^{n\times n}$.

\item Subsampling Matrix. Let $\Omega = \{\omega_1, \omega_2, \cdots, \omega_m\}$ be the subset of nodes drew independently from the set $\{1,2,\cdots n\}$ according to the sampling distribution $\bm{p}$. Define $M$ as the random subsampling matrix with the sampling distribution $\bm{p}$ such that

\[M_{ij} = 
\begin{cases} 
      1 & \text{ if } j = \omega_{i}\\
      0 & \text{otherwise}
   \end{cases}
\tag{2}\] 

for all $i \in \{1,2,\cdots,m\}$ and $j \in \{1,2,\cdots, n\}$.

\end{enumerate}

\medskip

From the $m$ samples obtained using the sampling method above, we can reconstruction all $\R$-concentrated signals accurately by solving the optimization problem 
\[\underset{\bm{f} \in \text{span}(\UR)}{\text{min}}\vert\vert P^{-1/2}_\Omega (M\bm{f}-\bm{y})\vert\vert_2, \tag{3}\] 
which estimates signal $\bm{f} \in \RR^n$ from $\bm{y} \in \RR^m$. 

Let $\epsilon, \delta \in (0,1)$. Let $\bm{f^*}$ be the solution of the problem (3). With probability at lease $1-\epsilon$, the following holds for all $\bm{f}\in \text{span}(U_{\R})$ and all $\bm{n}\in \RR^m: $

\[||x^* - x||_2 \leq \frac{2}{\sqrt{m(1-\delta)} ||P_{\Omega}^{-1/2} \bm{n}||_2} \tag{4}\]
provided that \[m \geq \frac{3}{\delta^2}\vv^k_{\bm{p}}\log(\frac{2k}{\epsilon}). \tag{5} \]

\end{thm}


{\color{blue}
\section{Illustrative Examples II: Approximate Calculations}
\begin{itemize}
\item Make sure to have some very large examples and show computation times
\end{itemize}
}







\end{document}