\documentclass[a4paper]{article}

%% Language and font encodings
\usepackage[english]{babel}
\usepackage[utf8x]{inputenc}
\usepackage[T1]{fontenc}

%% Sets page size and margins
\usepackage[a4paper,top=3cm,bottom=2cm,left=3cm,right=3cm,marginparwidth=1.75cm]{geometry}

%% Useful packages
\usepackage{listings}
\usepackage{algorithm}
\usepackage[noend]{algpseudocode}
\usepackage{float}
\usepackage{amsmath,bm,mathrsfs}
\usepackage{amsthm,amssymb}
\usepackage{graphicx}
%\usepackage[draft]{graphicx}
\usepackage[colorinlistoftodos]{todonotes}
\usepackage[colorlinks=true, allcolors=blue]{hyperref}

%% Operators
\newcommand{\R}{\mathcal{R}}
\newcommand{\RR}{\mathbb{R}}
\newcommand{\N}{\mathbb{N}}
\newcommand{\G}{\mathcal{G}}
\newcommand{\V}{\mathcal{V}}
\newcommand{\E}{\mathcal{E}}
\newcommand{\La}{\mathcal{L}}
\newcommand{\UR}{U_{\mathcal{R}}}
\newcommand{\vv}{\mathit{v}}

\theoremstyle{definition}
\newtheorem*{definition}{Definition}
\newtheorem*{thm}{Theorem}

\title{Notes}

\begin{document}
\maketitle

\newpage

\section{Notations}
Consider a weighted, undirected graph $\G = \{ \V, \E, W\}$, where $\V$ is the set $N$ vertices, $\E$ is the set of edges and $W$ is the weighted adjacency matrix. Let $D$ be the diagonal matrix of vertex degrees. Denote $\La$ as the graph Laplacian, where $\La = D-W$. Since $\La$ is real, symmetric and semi-definite, we can diagonalize it as $\La =U\Lambda U^*$, where $\Lambda$ is the diagonal matrix of its real eigenvalues $\lambda_0, \lambda_1,\cdots, \lambda_{n-1}$ and the columns $u_0, u_1 \cdots u_{n-1}$ of $U$ are corresponding orthonormal eigenvectors of $\La$. We use $\UR$ to denote the submatrix formed by taking the columns of $U$ associated with the Laplacian eigenvalues indexed by $\mathcal{R} \subseteq \{0,1,\cdots,n-1\}$. And we use $U_{\mathcal{S},\mathcal{R}}$ to denote the submatrix formed by taking the rows of $U_{\mathcal{R}}$ associated with the vertices indexed b the set $\mathcal{S} \subseteq \{1,2,\cdots, N\}$. $\bm{\delta_i} \in \RR^n$ denotes the $i$th column of the identity matrix $I \in \RR^{n\times n}.$

\medskip

\section{Definitions}
%\begin{definition}
%A signal $\bm{x} \in \RR^n$ defined on the nodes of the graph $\G$ is $k$-bandlimited with $k \in \N^+$ if $\bm{x} \in \text{span}(U_{\R})$ and $|\R| = k$.
%\end{definition}

\begin{definition}
A signal $\bm{f} \in \RR^n$ defined on the nodes of the graph $\G$ is $\R$-concentrated if $\bm{f} \in \text{span}(U_{\R})$.
\end{definition}

\begin{definition}
Let $\bm{p} \in \RR^{n}$ represent a sampling distribution on $\{1,2,\cdots n\}$. The graph weighted coherence of order $|\R| $ for the pair $(\G, \bm{p})$ is
$$ \vv^k_{\bm{p}} = \underset{1\leq i\leq n}{\text{ max }} \{\bm{p_i}^{-1/2} ||U_{\R}^T \bm{\delta_i} ||_2\}$$

\end{definition}

\section{Theorem}
\begin{thm}[\cite{puy}, Theorem 3.2]
Consider any graph $\G = \{\V, \E, W\}$ with its Laplacian $\La = U\Lambda U^*$. Notice that $\La$ is real, symmetric and positive semi-definite and thus the columns of U are orthonormal and its eigenvalues are non-negative. We propose the following sampling procedure:

\begin{enumerate}

\item Sampling Distribution. Let $\R \subseteq \{0,1,\cdots, n-1\}$. Define $\bm{p} \in \RR^n$ as the sampling distribution on vertices $\{1,2, \cdots n\}$  such that 

\[\bm{p_i} = \frac{||U^T_\R\bm{\delta_i}||^2_2}{|\R|} \text{, for } i = 1,2 ,\cdots, n. \tag{1}\]  
The sampling distribution $\bm{p}$ minimizes the graph weighted coherence. We associate the matrix $P = \text{diag}(\bm{p}) \in \RR^{n\times n}$.

\item Subsampling Matrix. Let $\Omega = \{\omega_1, \omega_2, \cdots, \omega_m\}$ be the subset of nodes drew independently from the set $\{1,2,\cdots n\}$ according to the sampling distribution $\bm{p}$. Define $M$ as the random subsampling matrix with the sampling distribution $\bm{p}$ such that

\[M_{ij} = 
\begin{cases} 
      1 & \text{ if } j = \omega_{i}\\
      0 & \text{otherwise}
   \end{cases}
\tag{2}\] 

for all $i \in \{1,2,\cdots,m\}$ and $j \in \{1,2,\cdots, n\}$.

\end{enumerate}

\medskip

From the $m$ samples obtained using the sampling method above, we can reconstruction all $\R$-concentrated signals accurately by solving the optimization problem 
\[\underset{\bm{f} \in \text{span}(\UR)}{\text{min}}\vert\vert P^{-1/2}_\Omega (M\bm{f}-\bm{y})\vert\vert_2, \tag{3}\] 
which estimates signal $\bm{f} \in \RR^n$ from $\bm{y} \in \RR^m$. 

Let $\epsilon, \delta \in (0,1)$. Let $\bm{f^*}$ be the solution of the problem (3). With probability at lease $1-\epsilon$, the following holds for all $\bm{f}\in \text{span}(U_{\R})$ and all $\bm{n}\in \RR^m: $

\[||x^* - x||_2 \leq \frac{2}{\sqrt{m(1-\delta)} ||P_{\Omega}^{-1/2} \bm{n}||_2} \tag{4}\]
provided that \[m \geq \frac{3}{\delta^2}\vv^k_{\bm{p}}\log(\frac{2k}{\epsilon}). \tag{5} \]

\end{thm}

\begin{thm}[Orthogonality Proof]

Let $\{\R_1, \R_2, \cdots, \R_M\}$ be $M$ partitions of the graph Laplacian eigenvalue indices $\{0, 1, \cdots,  N \}$ and $\{g_1, g_2, \cdots, g_M \}$ be filters defined on each of the $M$ bands. Then, the subspace spanned by $g_i(\La)$ is orthogonal to the subspace spanned by $g_j(\La)$ for $i \neq j$

\end{thm}

\begin{proof}


Notice that the $(i,j)$ entry of matrix $g(\La)$ can be represented as 

$$g(\La) (i,j) = [Ug(\Lambda)U^T] (i,j) = \sum_{k = 1}^{N} g(\lambda_k) U_k(i) U_k(j)$$

Now consider two filters $g_1$ and $g_2$. To show that two subspaces spanned are orthogonal, we show that the dot product of any two vectors is 0.
\begin{align*}
g_1(\La)(,k) \cdot g_2(\La)(,l)&=\ \  \sum_{j = 1}^{N} [\sum_{k = 1}^{N} g_1(\lambda_k) U_k(i) U_k(j) )( \sum_{l = 1}^{N} g_2(\lambda_l) U_l(i) U_l(j)) ] \\
& = \sum_{k = 1}^{N} \sum_{l = 1}^{N} g_1(\lambda_k) g_2(\lambda_l)  U_k(i) U_l(i) ( \sum_{j = 1}^{N}  U_k(j)  U_l(j))  \\
& = \sum_{k = 1}^{N} g_1(\lambda_k) g_2(\lambda_k)  U_k(i) U_k(i) ( \sum_{j = 1}^{N}  U_k(j)  U_k(j)) 
\end{align*}

Notice that $g_1(\lambda_k) g_2(\lambda_k)$ is always equal to $0$, since $g_1$ and $g_2$ are defined on disjoint sets of Laplacian eigenvalues.   

\end{proof}



%\begin{proof}

%$$g_1(\La) = Ug_1(\Lambda)U^T$$

%$$g_2(\La) = Ug_2(\Lambda)U^T$$

%Notice that $g_1(\lambda_k) g_2(\lambda_k)$ is always equal to $0$, since $g_1$ and $g_2$ are defined on disjoint sets of Laplacian eigenvalues.  
%\end{proof}
%\end{comment}



















\end{document}